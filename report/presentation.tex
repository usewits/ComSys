\documentclass{beamer}
\usepackage{amsmath, amsthm, amssymb}
\usepackage{graphicx}
\usepackage{booktabs}
\usepackage{geometry}
\usepackage{algpseudocode}
\usepackage{float}
\usepackage{caption}
\usepackage{animate}
\usepackage{tikz}

\renewcommand{\mod}{~\mathrm{mod}~}
\renewcommand{\div}{~\mathrm{div}~} 
\renewcommand{\vec}[1]{\mathbf{#1}}
\renewcommand{\phi}{\varphi}

\renewcommand{\(}{\left(}
\renewcommand{\)}{\right)}

\newcommand{\dt}{\Delta t}
\newcommand{\dx}{\Delta x}
\newcommand{\hl}[1]{\textbf{#1}}
\newcommand{\ceil}[1]{\lceil #1 \rceil}
\newcommand{\floor}[1]{\lfloor #1 \rfloor}
\newcommand{\half}[1]{\frac{{#1}}{2}}

\mode<presentation>{
    \usetheme{Warsaw}
    \setbeamertemplate{navigation symbols}{}
    \setbeamersize{text margin left=2em,text margin right=2em}
    \useoutertheme[subsection=false]{miniframes}
    \usecolortheme{rose}
}

\title[String Method]{The String Method} % short title / long title

\author{Abe Wits}
\institute[UU]
{
Utrecht University\\
\medskip
\textit{A.J.G.Wits@uu.nl}
}
\date{June 30, 2015}

\begin{document}

\begin{frame}
\titlepage
\end{frame}

\begin{frame}
\frametitle{Overview}
\tableofcontents
\end{frame}

\section{Introduction}
%\subsection{Transition Paths}

\begin{frame}
\frametitle{Transition Paths}
What does the path between two local optima look like?\\
~\\

\begin{figure}
\includegraphics[scale=0.3]{application.png}
\caption{B. Peters, A. Heyden, A. Bell, and A. Chakraborty\\ A growing string method for determining transition states:
Comparison to the nudged elastic band and string methods (2004)}
\end{figure}

Relevant to Material science, Quantum simulations, Microbiology, Nanotechnology, Chemistry and more.

\end{frame}

\begin{frame}
\frametitle{How to find Transition Paths?}
Monte Carlo Simulation\\
\begin{itemize}
\item[+] Easy to implement
\item[--] Extremely inefficient
\end{itemize}
~\\
More modern method needed:\\
\begin{itemize}
\item Nudged Electric Band (NEB) Method (1994)\\
\item String Method (2002)
\end{itemize}
~\\
Weinan E, Weiquing Ren and Eric Vanden-Eijnden\\
String method for the study of rare events (2002)
\end{frame}

\section{String Method}
%\subsection{Time-splitting scheme}
%Mention both gradient descent and reparametrisation
\begin{frame}
\frametitle{The Problem}
$$\gamma \dot{q} = -\nabla V(q)+\xi(t)$$
\begin{itemize}
\item $\gamma$ friction
\item $V$ a potential
\item $\xi$ some noise function
\end{itemize}
~\\
Let $A$ and $B$ two local minima of $V$\\
Wanted: $\phi^*$, the minimal energy path (MEP) from $A$ to $B$
$$(\nabla V)^\bot(\phi^*)=0$$
%Note that this also works on systems like Langevin equation
\end{frame}

\begin{frame}
\frametitle{The Solution}
Choose some initial $\phi$ from $A$ to $B$, let it evolve along $\nabla V$ to find $\phi^*$;
$$\phi_t=-[\nabla V(\phi)]^\bot+r\hat t$$
Where:
\begin{itemize}
\item $\hat t = {\phi_\alpha}/{|\phi_\alpha|}$\hspace{2em} the unit tangent vector along $\phi$
\item $(\nabla V)^\bot = \nabla V - (\nabla V \cdot \hat t)\hat t$\hspace{2em}  projection on $\hat t^\bot$ 
\item $r \equiv r(\alpha, t)$ \hspace{2em}a scalar field of Lagrange multipliers
\end{itemize}
~\\
It appears that E, Ren and Vanden-Eijnden do not actually use Lagrange multipliers to satisfy parametrisation constraints!%NOTE Should be comment?
\end{frame}

\begin{frame}
\frametitle{Splitting Method}
We choose the parametrisation such that for any $t$:
\begin{itemize}
\item $\alpha\in [0,1]$
\item $\phi(\alpha=0)=A$ 
\item $\phi(\alpha=1)=B$
\item $(|\phi_\alpha|)_\alpha = 0$ \hspace{3em} (Arc length is normalised)
\end{itemize}
~\\
We use a splitting method with two phases to satisfy all constraints: 
\begin{enumerate}
\item $\phi_t=-[\nabla V(\phi)]^\bot$ \hspace{3em} (Move towards $\phi^*$)
\item $(|\phi_\alpha|)_\alpha = 0$ \hspace{3em} (Keep arc length normalised)
\end{enumerate}
\end{frame}

\section{Implementation}

\begin{frame}
\frametitle{Discretization of $\phi$}
We choose $\phi_i = \phi(\alpha = \frac{i}{M})$, so that:
\begin{itemize}
\item $\phi$ becomes $\phi_0$, $\phi_1$, $\dots$, $\phi_M$
\item  $\phi_0 = A$
\item $\phi_M = B$
\end{itemize}
~\\
To obtain $\phi_\alpha$ we use a finite difference (FD) scheme:\\
~\\
\begin{columns}[c]
\column{.45\textwidth}
\begin{block}{Forward Difference}
$$(\phi_\alpha)_i = \frac{\phi_{i+1}-\phi_i}{\Delta}$$
\end{block}
\column{.45\textwidth}
\begin{block}{Central Difference}
$$(\phi_\alpha)_i = \frac{\phi_{i+1}-\phi_{i-1}}{2\Delta}$$
\end{block}%NOTE \Delta is strange here!
\end{columns}
~\\
~\\
We need $\phi_\alpha$ for $\hat t = \phi_\alpha / |\phi_\alpha|$, note that $\Delta$ cancels.
\end{frame}

\begin{frame}
\frametitle{Solving $\phi_t=-[\nabla V(\phi)]^\bot$}
$$(\phi_t)_i=-[\nabla V(\phi_i)]^\bot = \nabla V(\phi_i) - (\nabla V(\phi_i) \cdot \hat t_i)\hat t_i$$\\~\\
We choose a timestep $\dt$, let $t=n\dt$ for $n\in\{0, 1, \dots N\}$\\
$$\phi_i^{n+1} = \phi_i^n+\dt(\phi_t)_i^n$$
\begin{block}{To move towards $\phi^*$}
$$\phi_i^{n+1} = \phi_i^n+\dt\Big{(}\nabla V\(\phi_i^n\) - \(\nabla V\(\phi_i^n\) \cdot \hat t_i^n\)\hat t_i^n\Big{)}$$
\end{block}
\end{frame}

\begin{frame}
\frametitle{Solving $\phi_t=-[\nabla V(\phi)]^\bot$ more efficiently}
We can use Broyden's Method instead. We want to find $\phi$ such that:
$$F(\phi)=[\nabla V(\phi)]^\bot = 0$$
Now let $J_0^{-1}$ the inverse of the Jacobian of $F$ in the point $\phi_0$ (our initial $\phi$), denote $\Delta \phi = \phi_{n} - \phi_{n-1}$ and $\Delta F = F_n - F_{n-1}$.
\begin{block}{Calculate $\phi_n$}
$$\phi_{n}=\phi_{n-1} - J_{n-1}^{-1}F(x_{n-1})$$
\end{block}
Afterwards, we can update the inverse of the Jacobian directly:
\begin{block}{Calculate $J_n^{-1}$}
$$J_n^{-1} = J_{n-1}^{-1} + \frac{\Delta \phi_n - J_{n-1}^{-1}\Delta F_n}{\Delta \phi_n^{\top}J_{n-1}^{-1}\Delta F_n}\Delta \phi_n^\top J_{n-1}^{-1}$$
\end{block}

\end{frame}


%\subsection{Reparametrisation}
\begin{frame}
\frametitle{Satisfying $(|\phi_\alpha|)_\alpha=0$}
\begin{block}{$\mathcal{O}(n)$ approximate method (used by Eric Vanden-Eijnden)}
\begin{itemize}
\item Use piecewise linear interpolation
\item Let $L$ be the total length of this curve
\item Put $\phi_i$ at distance $L/M$ \textbf{along this curve}
\end{itemize}
\end{block}
~\\
\begin{block}{$\mathcal{O}(n\log \frac{L}{\epsilon})$ $\epsilon$-precision method (used by me)}
\begin{itemize}
\item Use piecewise linear interpolation
\item Try some distance $d$, set $||\phi_{i+1}-\phi_i|| = d \in [0, L/M]$
\item Improve iteratively until within $\epsilon$ precision (binary search)
\end{itemize}
\end{block}
~\\
...we will discuss experimental results later!
\end{frame}

%TODO write about alternate methods (Broydens)

\section{Numerical Experiments}
%\subsection{Peaks potential}
\begin{frame}
\frametitle{Peaks potential}
\vspace{-1em}
\begin{columns}[c]
\column{.35\textwidth}

The peaks potential:
\begin{itemize}
\item 2 dimensional
\item Smooth
\item Easy to visualise
\item Lots of peaks
\end{itemize}

\column{.65\textwidth}
\begin{center}
\includegraphics[scale=0.6]{peaks.png}
\end{center}
\end{columns}

\begin{align*}
V(x,y) & = 3 (1-x)^2 e^{-x^2-(y+1)^2}\\
&-10 e^{-x^2-y^2} \left(-x^3+\frac{x}{5}-y^5\right)\\
&-\frac{1}{3} e^{-(x+1)^2-y^2}\\
\end{align*}
\end{frame}

\begin{frame}
\frametitle{Peaks potential}
Some experiments:
\begin{itemize}
\item Comparison of central and forward difference (Animation 1)
\item Comparison of different timestep sizes (Animation 2)
\end{itemize}
~\\
Convergence with the string method:
\begin{center}
\includegraphics[scale=0.5]{peaksEnergy.png}
\end{center}

\end{frame}

%\subsection{Particles with Lennart Jones}
\begin{frame}
\frametitle{Particles with Lennard Jones (LJ) potential }
\vspace{-1.5em}
\begin{center}
\includegraphics[scale=0.35]{ljAB.png}
\end{center}

\begin{columns}[c]
\column{.5\textwidth}
Soft Lennard Jones potential:
$$V(r) = \(\frac{1}{r}\)^4-2\(\frac{1}{r}\)^2$$

``Hard'' Lennard Jones potential:
$$V(r) = \(\frac{1}{r}\)^{12}-2\(\frac{1}{r}\)^6$$

\column{.5\textwidth}
\includegraphics[scale=0.45]{ljplot.png}
\end{columns}

\end{frame}

\begin{frame}
\frametitle{Particles with Lennard Jones (LJ) potential}
\begin{tabular}{cc}
Soft LJ (Animation 3) & Hard LJ (Animation 4)\\
&\\
\includegraphics[scale=0.43]{SoftLJ200Energy.png} &
\includegraphics[scale=0.43]{HardLJ200Energy.png}\\
\includegraphics[scale=0.35]{SoftLJ200energy_vs_iteration.png} &
\includegraphics[scale=0.35]{HardLJ200energy_vs_iteration.png}
\end{tabular}
\end{frame}


%\subsection{Reparametrisation}
\begin{frame}
\frametitle{Reparametrisation}
\begin{tabular}{cc}
Vanden-Eijnden & Me \\
\includegraphics[scale=0.4]{reparamInterp1.png}&
\includegraphics[scale=0.4]{reparamCustom.png}\\
&
\includegraphics[scale=0.4]{reparamCustomPartial.png}
\end{tabular}

\end{frame}


%\subsection{Waves in solution}
%\begin{frame}%TODO write this

%\end{frame}
\begin{frame}
\frametitle{Discussion}
\begin{itemize}
\item Why does my reparametrisation function not always reach $\epsilon$ precision?
\item Provide faster implementation of my reparametrisation function
\end{itemize}
\end{frame}

\section{Beyond the String Method}
\begin{frame}
\frametitle{Beyond the String Method}
Improved versions of the Sting Method:
\begin{itemize}
\item Growing String Method (2004)
\item String Method in collective variables (aka Mean Forces) (2006)
\item On the fly String Method (2007)
%On-the-fly string method for minimum free energy paths calculation
%Luca Maragliano, Eric Vanden-Eijnden
\item Swarm of trajectories String Method (2008)
%Finding Transition Pathways Using the String Method with Swarms of Trajectories
%Albert C. Pan, Deniz Sezer, and Benoît Roux
%
%    \begin{itemize}
%    \item Faster in practice
%    \item Does not need initial string!
%    \end{itemize}
\end{itemize}
~\\
Goals in research:
\begin{itemize}
\item Faster convergence
\item No initial estimate required
\item Better performance in rough potentials
\end{itemize}
\end{frame}

\end{document} 

